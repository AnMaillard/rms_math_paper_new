We discuss briefly here Open Problem~\ref{op:freezing}, and more specifically how freezing could be established, 
or at least conjectured, from the second moment computation.
For the symmetric binary perceptron (SBP), freezing was conjectured in this way in~\cite{aubin2019storage}, before being established rigorously in~\cite{perkins2021frozen,abbe2022proof}.

\myskip 
Let us first fix some notations.
For the remainder of this section we assume that the margin $\kappa \in (0,2)$ is given and fixed.
We denote $\Sigma_n \coloneqq \{\pm 1\}^n$, and $\mcS_d$ the set of symmetric $d \times d$ matrices.
We let $\bH \coloneqq (\bW_1, \cdots, \bW_n)$, and denote $S(\bH) \coloneqq \{\eps \in \{\pm 1\}^n \, \textrm{s.t.} \, \|\sum_{i=1}^n \eps_i \bW_i\|_{\op} \leq \kappa \sqrt{n} \}$.
Recall that $Z_\kappa = |S(\bH)|$, see eq.~\eqref{eq:def_Zkappa}.
In this appendix only, we call $\bbP_0$ the law of $\bW_1, \cdots, \bW_n \iid \GOE(d)$, which was denoted $\bbP$ in the rest of the manuscript.

\begin{definition}[Freezing]\label{def:freezing}
    Let $\kappa \in (0,2)$, and $\tau = n/d^2$, such that $Z_\kappa \geq 1$ with probability $1 - \smallO(1)$ as $d \to \infty$ (i.e.\ we are in the SAT phase).
    The set $S(\bH)$ of solutions is said to exhibit freezing
    if there exists $q_c \in (0,1)$ such that the following holds:
    \begin{align*}
        \plim_{n,d \to \infty} \frac{1}{n} \log |\{\eps' \in S(\bH) \, : \, \eps \cdot \eps' \geq n q_c\}| = 0,
    \end{align*}
    where the limit is in probability over $\bH$ and $\eps \sim \Unif(S(\bH))$.
\end{definition}
\noindent
Definition~\ref{def:freezing} is a relatively weak form of freezing: for a typical solution $\eps$, the number of solutions with overlap to $\eps$ greater than $q_c$  
must be sub-exponential in $n$. In the SBP, one can show that $\eps$ is completely isolated, and thus the \emph{only} such solution.
In what follows, we describe the two key ingredients in the classical route to establish freezing, as was done e.g.\ for the SBP.

\myskip 
\textbf{Step 1: contiguity --}
Freezing of solutions is a property of $\Unif(S(\bH))$, the uniform measure over the set of solutions.
The classical approach to establish properties of this measure is to use a contiguity argument with a \emph{planted} version of the problem, 
which is often easier to study.
Recall that, for two sequences of probability distributions $\bbP_d$ and $\bbQ_d$ (that depend on $d$), we say that $\bbP_d$ is contiguous to $\bbQ_d$, denoted $\bbP_d \lhd \bbQ_d$, 
if for all sequences of events $A_d$:
\begin{align}\label{eq:contiguity}
(\bbQ_d(A_d) \to 0 \textrm{ as } d \to \infty)
 \Rightarrow 
(\bbP_d(A_d) \to 0 \textrm{ as } d \to \infty).
\end{align}

\myskip
In our setting, we define two models.
\begin{itemize}[leftmargin=*]
    \item \textbf{Random model} ($\bbP_{\ra}$): draw first $\bW_1, \cdots, \bW_n \iid \GOE(d)$, conditioned on $S(\bH) \neq \emptyset$.
    Draw then $\eps \sim \Unif(S(\bH))$.
    \item \textbf{Planted model} ($\bbP_{\pl}$): draw first $\eps \sim \Unif(\Sigma_n)$. Draw then $\bW_1, \cdots, \bW_n \iid \GOE(d)$, conditioned on 
    satisfying $\|\sum_i \eps_i \bW_i\|_\op \leq \kappa \sqrt{n}$.
\end{itemize}
$\bbP_\ra$ and $\bbP_\pl$ define two probability measures over $\mcS_d^n \times \Sigma_n$. In the planted distribution, 
we often denote $\eps = \eps^\star$.
In many problems, $\bbP_\pl$ turns out to be easier to study than $\bbP_\ra$, and moreover we have the general identity, 
for any $(\eps, \bH)$ such that $\eps \in S(\bH)$:
\begin{align}\label{eq:planted_random}
    \frac{\rd \bbP_{\pl}}{\rd \bbP_{\ra}}(\eps, \bH) &= \frac{Z_\kappa(\bH)}{\EE_0[Z_\kappa(\bH')]} \cdot \bbP_0(S(\bH') \neq 0),
\end{align}
When $\bbP_0(S(\bH') \neq 0) \to 1$ as $d \to \infty$, contiguity of $\bbP_{\pl}$ and $\bbP_{\ra}$ thus follows from concentration of $Z_\kappa$ on its average. 
More generally,
if for a given sequence $\alpha_d > 0$ one can establish that there exists $M > 0$ such that
\begin{align}\label{eq:concentration_fenergy}
%    \frac{1}{d^2} \log Z_\kappa = \lim_{d \to \infty} \frac{1}{d^2} \log \EE[Z_\kappa] + \mcO\left(\frac{\alpha_d}{d^2}\right),
    \lim_{d \to \infty}\bbP_0\left(\frac{Z_\kappa}{\EE[Z_\kappa]} \geq e^{-M\alpha_d}\right) &= 0,
\end{align}
then one can show that events that have probability $\exp(-\omega(\alpha_d))$ under $\bbP_{\pl}$ must have probability $\smallO(1)$ under $\bbP_\ra$~\citep{perkins2021frozen}.
In particular, if $\alpha_d = \mcO(1)$ then $\bbP_\pl \lhd \bbP_\ra$.
Unfortunately, the bound we establish in Proposition~\ref{prop:2nd_moment} does not imply a bound of the type of eq.~\eqref{eq:concentration_fenergy}.
Establishing concentration of $Z_\kappa$ in average-case matrix discrepancy also appears more complex than in the SBP, 
where the fluctuations of $Z_\kappa$ can be exactly characterized~\citep{abbe2022proof}:
whether this concentration holds in the whole SAT phase is also unclear, given the failure of the second-moment method in parts of the phase diagram, see Theorem~\ref{thm:fail_second_moment}.
For these reasons, we leave the establishment of contiguity as an open problem.
\begin{openquestion}[Contiguity]\label{op:contiguity}
    For $\kappa \in (0,2)$ and $\tau = n/d^2$ sufficiently large (such that $S(\bH) \neq \emptyset$ with high probability), are $\bbP_\ra$ and $\bbP_\pl$  
    contiguous as $d \to \infty$? Can one at least show a bound of the type of eq.~\eqref{eq:concentration_fenergy}?
\end{openquestion}

\myskip
\textbf{Step 2: is the planted solutions isolated ?}
For $l \in [n]$, denote $q_l \coloneqq 2l/n - 1 \in [-1,1]$.
A key point is that for any $q_0 = q(l_0) \in [-1,1]$:
\begin{align*}
    &\bbP_\pl(\exists \eps \in S(\bH) \backslash \{\eps^\star\} \, : \,  \eps \cdot \eps^\star \geq n q_0) \\
    &\aleq 
    \EE_\pl \left[\# \{\eps \in S(\bH) \backslash \{\eps^\star\} \, : \, \eps \cdot \eps^\star \geq n q_0\}\right], \\ 
    &\bleq \sum_{l \geq l_0} \binom{n}{l} \bbP_{\pl} \left[\left\|\sum_{i=1}^n \eps_i \bW_i\right\|_\op \leq \kappa\right], \hspace{20pt} (\textrm{for any } \eps \in \Sigma_n \textrm{ s.t. } \eps \cdot \eps^\star = n q_l), \\
    &\cleq \sum_{l \geq l_0}\binom{n}{l} \frac{2^n}{\EE_0[Z_\kappa]} \bbP_{0} \left[\|\bW\|_\op \leq \kappa \textrm{ and } \|q_l \bW + \sqrt{1-q_l^2} \bZ\|_\op \leq \kappa\right], \\
    &\dleq \sum_{l \geq l_0}\binom{n}{l} \frac{\bbP_{0} \left[\|\bW\|_\op \leq \kappa \textrm{ and } \|q_l \bW + \sqrt{1-q_l^2} \bZ\|_\op \leq \kappa\right]}{\bbP_{0} \left[\|\bW\|_\op \leq \kappa\right]}.
\end{align*}
where in $(\rm a)$ we used Markov's inequality, in $(\rm b)$ we used the invariance of the law of $\GOE(d)$, in $(\rm c)$ the definition of the planted distribution, and in $(\rm d)$ the expression of $\EE_0[Z_\kappa]$, see Section~\ref{sec:1st_moment}.
Given this derivation and, assuming contiguity with the planted model, for a given $\tau = \lim n / d^2$ in the SAT phase (i.e.\ such that $Z_k \geq 1$ with high probability), a sufficient condition for the freezing of solutions 
is that there exists $q_c \in (0,1)$ such that for all $q \in (q_c, 1)$: 
\begin{align}\label{eq:sufficient_cond_freezing}
    \lim_{d \to \infty} \frac{1}{d^2} \log \frac{\bbP_0 \left[\|\bW\|_\op \leq \kappa \textrm{ and } \|q \bW + \sqrt{1-q^2} \bZ\|_\op \leq \kappa\right]}{\bbP_0 \left[\|\bW\|_\op \leq \kappa\right]} + \tau H\left(\frac{1+q}{2}\right) < 0.
\end{align}
Unfortunately, the bounds we derive in Section~\ref{sec:2nd_moment} for the first term in eq.~\eqref{eq:sufficient_cond_freezing} become trivial as $q$ approaches $1$. 
We therefore leave an investigation of eq.~\eqref{eq:sufficient_cond_freezing} as an open question, which could be elucidated by a sharp second moment analysis (see Open Problem~\ref{op:sharp_2nd_mom}).
\begin{openquestion}[Second moment potential close to $q = 1$]
    Establish whether eq.~\eqref{eq:sufficient_cond_freezing} holds for $\tau = n /d^2 = \Theta(1)$ sufficiently large (as a function of $\kappa \in (0,2)$), and some $q_c < 1$.
\end{openquestion}