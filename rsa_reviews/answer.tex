\documentclass[a4paper,11pt]{article}
\usepackage[top=2.5cm, bottom=2.5cm, left=2cm, right=2cm]{geometry}
\usepackage{settings}
%\addbibresource{refs.bib}
\renewcommand{\section}[2]{}%Avoid bib title
\setlength{\abovecaptionskip}{0pt}
\setlength{\belowcaptionskip}{-10pt}

% paragraph formatting
\setlength{\parindent}{0pt}
\setlength{\parskip}{5.5pt}

% no page numbers
\pagestyle{empty}

\begin{document}

\begin{center}
{\bf \Large Answer to the referees}
\end{center}

I would like to thank the referees for their careful reading of the paper, 
their positive assessment, and their comments that will greatly help me to improve the presentation and the overall quality of the manuscript. 
I implemented all the minor comments and corrections suggested by the referees. 
I answer below to the more significant points raised, and I summarize the main changes implemented in the revised version.

\myskip 
For the referees' convenience, I highlighted in dark red the changes made in the revision.
Equation numbers and pages refer to the new revised version.

\myskip 
\noindent\textbf{Points raised by Reviewer~1.}
\begin{itemize}
    \item Following the referee's advice, I introduced in Section~1.3 a series of open questions arising from this work, 
    in order to make them more accessible to the reader. 
    \item I added a new Appendix~A, discussing the freezing of solutions and how this phenomenon 
    could be studied via the second moment analysis. 
    In particular, I highlight open problems concerning contiguity with planted models, 
    and the behavior of the second moment potential close to $q=1$. 
    \item In Figure~1, I added the region $\kappa > 2$, to clarify that in this case the problem is trivially satisfiable.
\end{itemize}
\myskip 
\noindent\textbf{Points raised by Reviewer~2.}
\begin{itemize}
    \item Below Proposition~2.1 I clarified why I study the conditioned GOE law directly: 
    the conditioned ensemble is indeed a $\beta$-ensemble (with a singular potential enforcing the spectral norm constraint), 
    which makes the large deviation analysis more straightforward. 
    While one could alternatively use the large deviations for unconditioned GOE matrices, 
    this approach requires extra care because the set of probability measures supported in $[-\kappa, \kappa]$ has empty interior under the weak topology. 
    I had originally derived the result using this large deviations principle for unconditioned GOE matrices, and this derivation is now included in Appendix~A for completeness. 
    I chose to present the conditioned approach in the main text because it directly leverages existing results on $\beta$-ensembles (see Proposition~2.3). 
    As the reviewer points out, this
    also generalizes to norm-constrained matrix ensembles beyond the GOE. 
    I modified the discussion in the main text to clarify this point.
    \item Regarding Lemma~3.5, I agree the bound on the variance is what is used in the rest of the proof. 
    However --- if I understood correctly the reviewer's comment --- I am not convinced that the fact that the sub-Gaussian norm directly controls the variance 
    deserves a separate lemma: I instead moved this fact as a remark below Lemma~3.7.
\end{itemize}
Finally, I removed on page~11 a couple of sentences related to the possibility of applying 
the second moment method in a conditional manner to yield a sharp SAT/UNSAT threshold, 
as I believe these statements were somehow speculative. 

\end{document}